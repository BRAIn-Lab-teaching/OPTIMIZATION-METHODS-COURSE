\begin{center}
    \textbf{Домашнее задание 9, двойственность по Лагранжу}
\end{center}

\section*{Основная часть}

\begin{enumerate}[label=\textbf{Задача \arabic*.}]

    \item (1 балл) Найдите двойственную задачу к задаче
    \begin{equation*}
        \begin{aligned}
            \max_{x_1, x_2, x_3} & x_1 + 2x_2 + 3x_3 \\
            \text{s.t.} \quad & 4x_1 + 5x_2 + 6x_3 \leq 7, \\
                              & 8x_1 + 9x_2 + 10x_3 = 11, \\
                              & x_1 \geq 0.
        \end{aligned}
    \end{equation*}

    \item (1 балл) Найдите двойственную задачу к задаче
    \begin{equation*}
        \begin{cases}
            -9x_1 + 8x_2+10x_3 \rightarrow \min \\
            -4x_1 + 4x_3 \leq 1 \\
            5x_1 + x_2 + 7x_3 = 0 \\
            -8x_1 - 7x_2 + 3x_3 \geq 2 \\
            6x_1 + 3x_2 + 8x_3 \leq 3 \\
            x_1 \geq 0 \\
            x_2 \geq 0.
        \end{cases}
    \end{equation*}

    \item (1 балл) В терминах двойственной задачи сформулировать условия несовместности системы
    \begin{equation*}
        \begin{cases}
            A_1 x \leq b_1 \\
            A_2 x = b_2.
        \end{cases}
    \end{equation*}
    
    \textit{Указание.} К данной системе можно приписать "искусственную"\ целевую функцию, у которой оптимум гарантированно достигается и один и тот же при любых переменных, удовлетворяющих ограничениям. Тем самым, задача формально превратится в задачу оптимизации. Далее, можно построить двойственную к ней и, пользуясь довольно известным результатом, сказать, что исходная система совместна тогда и только тогда, когда совместна двойственная. Остается только заметить, что задача линейного программирования несовместна, если либо не существует хотя бы одного набора переменных, удовлетворяющих ее ограничениям, либо такие наборы могут "улетать"\ на бесконечность, тем самым экстремум не будет достижим.

    \item (2 балла) Рассмотрим задачу линейного программирования
    \begin{equation}
    \begin{cases}
        c_1x_1 + c_2x_2 \rightarrow \max \\
        A_{11}x_1 + A_{12}x_2 \leq b_1 \\
        A_{21}x_1 + A_{22}x_2 = b_2 \\
        x_1 \leq 0,
        \label{eq:linprog_classic}
    \end{cases}
    \end{equation}
    
    \begin{enumerate}
        \item (0.5 балла) Найдите двойственную к ней задачу.
        \item (0.75 балла) Пользуясь результатами прошлого пункта, покажите, что если $(x_1, x_2)$ удовлетворяет ограничениям ЗЛП (\ref{eq:linprog_classic}), а $(u_1, u_2)$ удовлетворяет ограничениям двойственной к ЗЛП (\ref{eq:linprog_classic}) задачи, то выполняется
        \begin{equation*}
            b_1 u_1 + b_2 u_2 \geq c_1 x_1 + c_2 x_2.
        \end{equation*}
        \item (0.75 балла) Покажите, что если выполнены условия прошлого пункта, то для того, чтобы $(x_1, x_2)$ был решением ЗЛП (\ref{eq:linprog_classic}), а $(u_1, u_2)$ был решением двойственной к ЗЛП (\ref{eq:linprog_classic}) задачи, необходимо и достаточно, чтобы значения функционалов этих задач совпадали.
        
    \end{enumerate}
        
\end{enumerate}


\section*{Дополнительная часть}

\begin{enumerate}[label=\textbf{Задача \arabic*.}]

    \item (2 балла) Построите двойственную задачу для следующей задачи оптимизации:
    \begin{align*}
            \min_{x} \ -\sum\limits_{i=1}^m \log (b_i - a_i^Tx)
    \end{align*}
    с областью определения $\{x \in \mathbb{R}^d \ | \ a_i^Tx < b_i \ \forall i = 1\ldots m\}$.
    
    \textit{Hint:} Сначала введите дополнительные переменные $y_i$ и ограничения $y_i = b_i - a_i^Tx$.

    \item (3 балла) Рассмотрим задачу следующего вида: 
    \begin{align*}
         \min_{x} \ &c^Tx\\
         \text{s.t. } &Ax \preceq b,\\
         &x_i \in \{0, 1\}, \ \forall i = 1, \ldots, d.
     \end{align*}
     Эта задачу довольно трудно решать, поэтому есть две релаксации, которые помогают построить нижнюю оценку на оптимальное значение исходной задачи.
     Рассмотрим следующие две задачи оптимизации, которые схожи с начальной:
     \begin{align*}
         \min_{x} \ &c^Tx\\
         \text{s.t. } &Ax \preceq b,\\
         &0 \leq x_i \leq 1, \ \forall i = 1, \ldots, d
     \end{align*}
     и
     \begin{align*}
         \min_{x} \ &c^Tx\\
         \text{s.t. } &Ax \preceq b,\\
         &x_i(1 - x_i) = 0, \ \forall i = 1, \ldots, d.
     \end{align*}
     Первая задача называется LP-релаксацией исходной задачи, и, как следует из ее записи, она дает нижнюю оценку на наше исходное оптимальное значение, а вторая задача является прямой перезаписью исходной задачи.


    \begin{enumerate}
        \item (1 балл) Выпишите двойственные задачи к LP-релаксации и ко второй задаче.
    
        \item (2 балла) Двойственная задача ко второй задаче называется Лагранжевой релаксацией. Как следствие, она тоже дает нижнюю оценку на оптимальное значение исходной задачи. Покажите, что нижние оценки, которые вытекают из LP-релаксации и Лагранжевой релаксации, совпадают.
    \end{enumerate}

\end{enumerate}