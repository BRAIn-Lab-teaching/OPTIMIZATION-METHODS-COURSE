\documentclass[a5paper,twoside,russian,8pt]{article}
\usepackage[intlimits]{amsmath}
\usepackage{amsthm,amsfonts}
\usepackage{amssymb}
\usepackage{mathrsfs}
\usepackage[final]{graphicx,epsfig}
\usepackage{indentfirst}
\usepackage[utf8]{inputenc}
\usepackage[T2A]{fontenc}
\usepackage[english]{babel}
\usepackage[usenames]{color}
\usepackage{hyperref}
\usepackage{wasysym}
\usepackage{enumitem}
\RequirePackage{enumitem}
\renewcommand{\alph}[1]{\asbuk{#1}}
\setenumerate[1]{label=\alph*), fullwidth, itemindent=\parindent, listparindent=\parindent} 
\setenumerate[2]{label=\arabic*), fullwidth, itemindent=\parindent, listparindent=\parindent, leftmargin=\parindent}
\usepackage{mathtools}

\hoffset=-10.4mm \voffset=-12.4mm \oddsidemargin=5mm \evensidemargin=0mm \topmargin=0mm \headheight=0mm \headsep=0mm
\textheight=174mm \textwidth=113mm

\newcommand{\bfv}{\mathbf}

\def \e {\varepsilon}

\def \ZZ {\mathbb Z}
\def \FF {\mathbb F}
\def  \R {\mathbb R}
\def \QQ {\mathbb Q}
\def \NN {\mathbb N}
\def \PP {\mathbb P}
\def \EE {\mathbb E}
\def \DD {\mathbb D}
\def \CC {\mathbb C}
\def \II {\mathbb I}


\def\Ss{\mathcal{S}^n} 
\def\Pp{\mathcal{P}^{n}}
\def\cA{{\cal A}}
\def\cB{{\cal B}}
\def\cD{{\cal D}}
\def\cC{{\cal C}}
\def\cQ{{\cal Q}}
%\def\R{{\cal R}}
\def\cM{{\cal M}}
\def\cN{{\cal N}}
\def\cT{{\cal T}}
\def\cP{{\cal P}}
\def\cF{{\cal F}}
\newcommand{\St}{\mathbb{S}}
\def\la{\langle}
\def\ra{\rangle}

\DeclareMathOperator*{\argmax}{arg\,max}
\DeclareMathOperator*{\argmin}{arg\,min}
\DeclareMathOperator*{\dom}{dom}

\newcommand{\Tr}{\operatorname{Tr}}
% Блочная матрица

\renewcommand{\Re}{\mathrm{Re}\,}
\renewcommand{\Im}{\mathrm{Im}\,}
\def\<{\langle}
\def\>{\rangle}




\begin{document}
\selectlanguage{russian}
\begin{center}
    \textbf{Домашнее задание 5}
\end{center}
\begin{center}
    \textbf{Deadline - 18.10.2024 в 23:59}
\end{center}

\section*{Основная часть}
\begin{enumerate}[label=\textbf{Задача \arabic*.}]
\subsection*{Выпуклые множества}
\item Проверьте, являются ли выпуклыми множества 
    \begin{enumerate}
        \item (0.5 балла) $S = \{ x_1 \in \mathbb{R}, x_2 \in \mathbb{R} \mid x_1 > 0, x_2 > 0, x_1 x_2 \geq 1 \}$.

        \item (0.5 балла) $S = \{ x \in \mathbb{R}^d \mid x_1 \leq x_2 \leq \ldots \leq x_d \}$.

        \item (1 балл) $S = \{ x \in \mathbb{R}^d \mid  \| x - a\|_2 \leq \| x - b\|_2 \}$, где $a\neq b \in \mathbb{R}^d$.
    \end{enumerate}
    
    \item Пусть $ S \subseteq \mathbb{R}^d$ и пусть $\|\cdot\|$ -- норма на $\mathbb{R}^d$.
    \begin{enumerate}
        \item (1 балл) Для $a \geq 0$ определим множество $S_a$ как:
        \begin{align*}
            S_a = \{x \mid \text{dist}(x, S) \leq a \},
        \end{align*}
        где 
        \begin{align*}
            \text{dist}(x, S) = \inf_{y \in S} \| x - y \|.
        \end{align*}
        Множество $S_a$ называется расширенным на $a$ относительно $S$. Докажите, что если $S$ выпукло, то $S_a$ также выпукло.
        \item (1 балл) Для $a \geq 0$ определим множество $S_{-a}$ как:
        \begin{align*}
            S_{-a} = \{x \mid B(x, a) \subset S\},
        \end{align*}
        где $B(x, a)$ - открытый шар (в норме $\| \cdot \|$) с центром в $x$ и радиусом $a$. Множество $S_{-a}$ называется суженным на $a$ относительно $S$. Докажите, что если $S$ выпукло, то $S_{-a}$ также выпукло.
    \end{enumerate}

    \item (1 балл) Пусть дано множество $X \subseteq \mathbb{R}^d$ и $x^0 \in X$. Докажите, что множество
    \begin{align*}
        K(X, x^0)=\left\{ y \in\mathbb{R}^d \mid y^T x^0 \geq y^T x \text{ for all } x \in X\right\}
    \end{align*}
    является выпуклым конусом.
\end{enumerate}

\subsection*{Выпуклые функции}
\begin{enumerate}[label=\textbf{Задача \arabic*.}]
    \item (1 балл) Пусть дана функция $f: \mathbb{R}^2 \to \mathbb{R}$. Выясните является ли она выпуклой, если $f(x) = x_1^2 x_2^2$.
    
    \item (1.5 балла) Пусть дана функция $f: \mathbb{R}^d \to \mathbb{R}$. Выясните является ли функция выпуклой/$\mu$-сильно выпуклой, если $f(x) = \sum\limits_{i=1}^{d} x_i^4$. В случае $\mu$-сильной выпуклости нужно найти и $\mu$.
    
    \item (1.5 балла) Пусть дана функция $f: \mathbb{S}^d \to \mathbb{R}$. Здесь $\mathbb{S}$ -- симметричные матрицы. Выясните является ли функция выпуклой/вогнутой, если
    \begin{enumerate}
        \item $f(X) = \lambda_{\max}(X)$
        \item $f(X) = \lambda_{\min}(X)$
    \end{enumerate}

    \item (1 балл) Пусть $f : \operatorname{dom} f \rightarrow \mathbb{R}$ -- функция с областью определения $\operatorname{dom} f \subseteq \mathbb{R}^d$. Покажите, что $f$ выпукла если и только если ее сужение на любую прямую выпукло. Формально это будет значить, что для любых $x_0 \in \operatorname{dom} f,u\in\mathbb{R}^d$, функция
$g: t\mapsto f(x_0 + tu)$
выпукла на  $\operatorname{dom} g:= \{{t\in\mathbb{R} : x_0 + tu\in \operatorname{dom} f}\}$.
\end{enumerate}

\newpage
\section*{Дополнительная часть}
\subsection*{Выпуклые множества}
\begin{enumerate}[label=\textbf{Задача \arabic*.}]    
    \item (1 балл) Назовем множество $X \subseteq \mathbb{R}^d$ "средневыпуклым", если для любых его элементов $x$ и $y$ их середина также принадлежит $X$, т.е. $\frac{x + y}{2} \in X$. Докажите, что для замкнутых множеств "средневыпуклость" равносильна выпуклости.

    \item (1.25 балла) Пусть $X = \{x_1, \ldots, x_{d+2}\}$ - множество из $d + 2$ точек в $\mathbb{R}^d$. Покажите, что $X$ можно разбить на два подмножества $S$ и $T = X \setminus S$ таким образом, что пересечение их выпуклых оболочек (см. определение в Пособии на странице 160) не пусто.

    \item Проверьте, верны ли следующие утверждения. Свою точку зрения объясните.
    \begin{enumerate}
        \item (1.25 балла)
        Проекция выпуклого множества на любое подпространство тоже выпукла.

        \textit{Пояснение}. Проекцией на множество $\mathcal{X}$  называется  $\Pi_{\mathcal{X}}(x) := \argmin\limits_{y \in \mathcal{X}} \frac{1}{2} \|x - y\|$.

        \item (1.5 балла)
        Если проекция на любое \emph{собственное} (не совпадающее со всем пространством) подпространство выпукла, то и изначальное множество выпукло?
    \end{enumerate}
\end{enumerate}

\subsection*{Выпуклые функции}
\begin{enumerate}[label=\textbf{Задача \arabic*.}]

    \item (1 балл) Докажите, что для всех $p, q \in \{ x \in \mathbb{R}^d \mid x_i \geq 0, \sum\limits_{i=1}^d x_i = 1\}$ справедливо следующее утверждение

    $$
        \sum_{i=1}^d \ln\left( \frac{p_i}{q_i}\right) p_i \geq 0.
    $$

    \item (1 балл) Пусть $g: \mathbb{R}_+ \to \mathbb{R}_+$ -- выпукла, $g(0) = 0$. Определим 
    $$f(x) = \frac{1}{x}\int_0^xg(t)dt,~ x > 0$$

    Покажите, что $f(x)$ -- тоже выпукла.

    \item (1.5 балла) Выясните является ли функция $f: \mathbb{S}^d_{++} \to \mathbb{R}$ выпуклой/вогнутой, если $f(X) = \text{Tr}(X^{-1})$.

    \item (1.5 балла) Воспользовавшись неравенством Йенсена для выпуклой на $\mathbb{R}_{++}$ функции $f(x) = -\ln{x}$, докажите неравенство Гёльдера:
    \begin{align*}
        \sum\limits_{i=1}^d x_i y_i \le \left( \sum\limits_{i=1}^d \vert x_i\vert ^p\right)^{1/p} \left( \sum\limits_{i=1}^d \vert y_i\vert^q\right)^{1/q}
    \end{align*}
    для $p >1,\ \dfrac{1}{p} + \dfrac{1}{q} = 1$. $\mathbb{R}_{++}$ -- положительные действительные числа.
\end{enumerate}
\end{document}