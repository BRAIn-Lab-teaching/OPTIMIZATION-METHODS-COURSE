\documentclass[a5paper,twoside,russian,8pt]{article}
\usepackage[intlimits]{amsmath}
\usepackage{amsthm,amsfonts}
\usepackage{amssymb}
\usepackage{mathrsfs}
\usepackage[final]{graphicx,epsfig}
\usepackage{indentfirst}
\usepackage[utf8]{inputenc}
\usepackage[T2A]{fontenc}
\usepackage[english]{babel}
\usepackage[usenames]{color}
\usepackage{hyperref}
\usepackage{wasysym}
\usepackage{enumitem}
\RequirePackage{enumitem}
\renewcommand{\alph}[1]{\asbuk{#1}}
\setenumerate[1]{label=\alph*), fullwidth, itemindent=\parindent, listparindent=\parindent} 
\setenumerate[2]{label=\arabic*), fullwidth, itemindent=\parindent, listparindent=\parindent, leftmargin=\parindent}
\usepackage{mathtools}

\hoffset=-10.4mm \voffset=-12.4mm \oddsidemargin=5mm \evensidemargin=0mm \topmargin=0mm \headheight=0mm \headsep=0mm
\textheight=174mm \textwidth=113mm

\newcommand{\bfv}{\mathbf}

\def \e {\varepsilon}

\def \ZZ {\mathbb Z}
\def \FF {\mathbb F}
\def  \R {\mathbb R}
\def \QQ {\mathbb Q}
\def \NN {\mathbb N}
\def \PP {\mathbb P}
\def \EE {\mathbb E}
\def \DD {\mathbb D}
\def \CC {\mathbb C}
\def \II {\mathbb I}


\def\Ss{\mathcal{S}^n} 
\def\Pp{\mathcal{P}^{n}}
\def\cA{{\cal A}}
\def\cB{{\cal B}}
\def\cD{{\cal D}}
\def\cC{{\cal C}}
\def\cQ{{\cal Q}}
%\def\R{{\cal R}}
\def\cM{{\cal M}}
\def\cN{{\cal N}}
\def\cT{{\cal T}}
\def\cP{{\cal P}}
\def\cF{{\cal F}}
\newcommand{\St}{\mathbb{S}}
\def\la{\langle}
\def\ra{\rangle}

\DeclareMathOperator*{\argmax}{arg\,max}
\DeclareMathOperator*{\argmin}{arg\,min}
\DeclareMathOperator*{\dom}{dom}

\newcommand{\Tr}{\operatorname{Tr}}
% Блочная матрица

\renewcommand{\Re}{\mathrm{Re}\,}
\renewcommand{\Im}{\mathrm{Im}\,}
\def\<{\langle}
\def\>{\rangle}

\begin{document}
\selectlanguage{russian}

\section{Домашнее задание 1, основная часть}

\begin{center}
    \textbf{Deadline - 20.09.2024 в 23:59}
\end{center}

В этой части используется следующие обозначения:

$\mathbb{R}_{++}$ - положительные вещественные числа

$I_n$ - матрица с единицами на диагонали (вне диагонали 0)

$A \in \mathbb{S}^n \quad\Longleftrightarrow \quad A= A^\top$

$A \in \mathbb{S}^n_+ \quad\Longleftrightarrow \quad A \in \mathbb{S}^n ; \quad  \forall x: \quad  x^\top Ax \geq 0$

$A \in \mathbb{S}^n_{++}\quad \Longleftrightarrow \quad A \in \mathbb{S}^n ; \quad \forall x \neq 0: \quad  x^\top Ax > 0$

Норма Фробениуса для матрицы $A \in \mathbb{R}^{n \times n}$ определяется как $||A||_F = \sqrt{\sum_{i=1}^n \sum_{j=1}^n A^2_{ij}}$

Для матриц скалярное произведение определено как $\langle X, Y \rangle := \text{Tr}(X^\top Y)$

\begin{enumerate}[label=\textbf{Задача \arabic*.}]

    \item Пусть $f$ -- одна из следующих функций:
    \begin{enumerate}
        \item (1 балл) $f : E \to \R$ -- функция $f(t) := \det(A - t I_n)$, где $A \in \R^{n \times n}$, $E := \{ t \in \R : \det(A - t I_n) \neq 0 \}$.
        \item (1.5 балла) $f : \R_{++} \to \R$ -- функция $f(t) := \| (A + t I_n)^{-1} b \|^2$, где $A \in \St^n_{++}$, $b \in \R^n$.
    \end{enumerate}
    Для каждого из указанных вариантов вычислите первую и вторую производные $f'(t)$ и $f''(t)$.

    \item Пусть $f$ -- одна из следующих функций:
    \begin{enumerate}
        \item (2 балла) $f : \R^n \to \R$ -- функция $\displaystyle f(x) := \frac{1}{2} \| x x^T - A \|_F^2$, где $A \in \St^n$.
        \item (2.5 балла) $f : \R^n \setminus \{ 0 \} \to \R$ -- функция $\displaystyle f(x) := \langle x, x \rangle^{\langle x, x \rangle}$.
    \end{enumerate}
    Для каждого из указанных вариантов вычислите градиент $\nabla f$ и гессиан $\nabla^2 f$ (относительно стандартного скалярного произведения в пространстве $\R^n$).

    \item Для каждой из следующих функций $f$ покажите, что вторая производная $d^2 f$ является знакоопределенной (как квадратичная форма) и установите ее знак:
    \begin{enumerate}
        \item (3 балла) $f : \St^n_{++} \to \R$ -- функция $f(X) := \langle X^{-1}, A \rangle$, где $A \in \St^n_+$.
    \end{enumerate}
        
\end{enumerate}


\section{Домашнее задание 1, дополнительная часть}

\begin{center}
    \textbf{Deadline - 20.09.2024 в 23:59}
\end{center}


В этой части используется следующие обозначения:

$\mathbb{R}_{++}$ - положительные вещественные числа

$A \in \mathbb{S}^n \quad\Longleftrightarrow \quad A= A^\top$

$A \in \mathbb{S}^n_{++}\quad \Longleftrightarrow \quad A \in \mathbb{S}^n ; \quad \forall x \neq 0: \quad  x^\top Ax > 0$

Норма Фробениуса для матрицы $A \in \mathbb{R}^{n \times n}$ определяется как $||A||_F = \sqrt{\sum_{i=1}^n \sum_{j=1}^n A^2_{ij}}$

\begin{enumerate}[label=\textbf{Задача \arabic*.}]

   \item Пусть $f$ -- одна из следующих функций:
    \begin{enumerate}
        \item (1.5 балл) $f : \R^n \to \R$ -- функция $\displaystyle f(x) := \frac{1}{2} \sum\limits_{i=1}^n\sum\limits_{j=1}^nx_ix_j$
        \item (1.5 балла)$f : \R^n \setminus \{ 0 \} \to \R$ -- функция $\displaystyle f(x) := \frac{\langle A x, x \rangle}{ \|x\|^2 }$, где $A \in \St^n$.
        \item (2 балла) $f: \R^n \to \R$ -- функция $f(x) := \ln(\sum_{i=1}^m e^{\langle a_i, x \rangle})$, где $a_1, \dots, a_m \in \R^n$. 
        Эта функция называется \href{https://en.wikipedia.org/wiki/LogSumExp}{LogSumExp}, и она используется для гладкого приближения $\max(Ax)$. Подумайте, почему так. Также см. функцию softmax с семинара.
    \end{enumerate}
    Для каждого из указанных вариантов вычислите градиент $\nabla f$ и гессиан $\nabla^2 f$ (относительно стандартного скалярного произведения в пространстве $\R^n$).

    \item Для каждой из следующих функций $f$ покажите, что вторая производная $d^2 f$ является знакоопределенной (как квадратичная форма) и установите ее знак:
    \begin{enumerate}
        \item (2 балла)$f : \R^n_{++} \to \R$ -- функция $f(x) := \prod_{i=1}^n x_i^{\alpha_i}$, где $\alpha_1, \dots, \alpha_n \geq 0$, $\sum_{i=1}^n \alpha_i = 1$. (\emph{Hint}: log derivative trick)
        \item (2 балл)$f : \R^n_{++} \to \R$ -- функция $f(x) := \left( \sum_{i=1}^n x_i^p \right)^{1/p}$, где $p < 1$, $p \neq 0$.
        \item (1 балла)$f : \St^n_{++} \to \R$ -- функция $f(X) := (\det(X))^{1/n}$.
    \end{enumerate}
    (\emph{Hint}: В некоторых пунктах могут оказаться полезными неравенства Коши--Буняковского и Йенсена.)
        
\end{enumerate}

\end{document}