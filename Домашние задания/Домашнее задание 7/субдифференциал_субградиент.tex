\documentclass[a5paper,twoside,russian,8pt]{article}
\usepackage[intlimits]{amsmath}
\usepackage{amsthm,amsfonts}
\usepackage{amssymb}
\usepackage{mathrsfs}
\usepackage[final]{graphicx,epsfig}
\usepackage{indentfirst}
\usepackage[utf8]{inputenc}
\usepackage[T2A]{fontenc}
\usepackage[english]{babel}
\usepackage[usenames]{color}
\usepackage{hyperref}
\usepackage{wasysym}
\usepackage{enumitem}
\RequirePackage{enumitem}
\renewcommand{\alph}[1]{\asbuk{#1}}
\setenumerate[1]{label=\alph*), fullwidth, itemindent=\parindent, listparindent=\parindent} 
\setenumerate[2]{label=\arabic*), fullwidth, itemindent=\parindent, listparindent=\parindent, leftmargin=\parindent}
\usepackage{mathtools}

\hoffset=-10.4mm \voffset=-12.4mm \oddsidemargin=5mm \evensidemargin=0mm \topmargin=0mm \headheight=0mm \headsep=0mm
\textheight=174mm \textwidth=113mm

\newcommand{\bfv}{\mathbf}

\def \e {\varepsilon}

\def \ZZ {\mathbb Z}
\def \FF {\mathbb F}
\def  \R {\mathbb R}
\def \QQ {\mathbb Q}
\def \NN {\mathbb N}
\def \PP {\mathbb P}
\def \EE {\mathbb E}
\def \DD {\mathbb D}
\def \CC {\mathbb C}
\def \II {\mathbb I}


\def\Ss{\mathcal{S}^n} 
\def\Pp{\mathcal{P}^{n}}
\def\cA{{\cal A}}
\def\cB{{\cal B}}
\def\cD{{\cal D}}
\def\cC{{\cal C}}
\def\cQ{{\cal Q}}
%\def\R{{\cal R}}
\def\cM{{\cal M}}
\def\cN{{\cal N}}
\def\cT{{\cal T}}
\def\cP{{\cal P}}
\def\cF{{\cal F}}
\newcommand{\St}{\mathbb{S}}
\def\la{\langle}
\def\ra{\rangle}

\DeclareMathOperator*{\argmax}{arg\,max}
\DeclareMathOperator*{\argmin}{arg\,min}
\DeclareMathOperator*{\dom}{dom}

\newcommand{\Tr}{\operatorname{Tr}}
% Блочная матрица

\renewcommand{\Re}{\mathrm{Re}\,}
\renewcommand{\Im}{\mathrm{Im}\,}
\def\<{\langle}
\def\>{\rangle}




\begin{document}
\selectlanguage{russian}
\begin{center}
    \textbf{Домашнее задание 7, субдифференциал и субградиент}
\end{center}
\begin{center}
    \textbf{Deadline - 01.11.2024 в 23:59}
\end{center}


\section*{Основная часть}
\begin{enumerate}[label=\textbf{Задача \arabic*.}]
    \item (1 балл) Пусть функция $f:\mathbb{R} \to \mathbb{R}$ задана следующим образом $f(x) = \max\{-x, x, x^2\}$. Найдите субдифференциал данной функции $\partial f(x)$.
    
    \item (0.5 балла) Найдите $\partial f(x)$, если $f(x) = \text{ReLU}(x) = \max \{0, x\}.$
    
    \item (1.5 балла) Пусть $f(x) = \|x\|_\infty$. Докажите, что $$ \partial f(0) = \textbf{conv}\{\pm e_1, \ldots , \pm e_n\}, $$ где $e_i$ это $i$-тый вектор канонического базиса (т.е. столбец единичной матрицы).
    
    \item (1 балл) Пусть функция $f:\mathbb{R} \to \mathbb{R}$ задана следующим образом $f(x) = | x - 2| + |x + 2| + |x - 1|$. Найдите субдифференциал данной функции $\partial f(x)$.
    
    \item (1 балл) Пусть функция $f: \mathbb{R}^n \to \mathbb{R}$ задана следующим образом $f(x) = \exp\left(\|Ax - b\|_p\right)$, где $A \in \mathbb{R}^{m \times n}$, $b \in \mathbb{R}^m$, $p \in [1; +\infty]$. Найдите субдифференциал $\partial f(x)$.
\end{enumerate}

\section*{Дополнительная часть}
\begin{enumerate}[label=\textbf{Задача \arabic*.}]
    \item (1 балл) Пусть $f: \mathbb{R}^n \to \mathbb{R}$ есть индикаторная функция следующего множества $$\mathcal{B}_{\|\cdot\|}(0, 1) = \left\{x : \| x\|_p \leq 1\right\},$$
    где $p \in [1; +\infty]$. Найдите субдифференциал $\partial f(x)$. 
    
    \item (1.5 балла) Пусть $f: S \to \mathbb{R}$ - функция, определенная на множестве $S$ из Евклидова пространства $E$. Пусть $x_0 \in S$ и пусть $f^*: S_{*} \to \mathbb{R}$ - сопряженная функция, где $S_*$ из сопряженного пространства $E^*$. Покажите,что $$\partial f(x) = \left\{g \in S_{*} : \langle g, x\rangle = f^*(g) + f(x)\right\}$$
    
    \item (1 балл) Пусть $\lambda_{\max}: \mathbb{S}^d \to \mathbb{R}$ - функция максимального собственного значения, заданная на $\mathbb{S}^d$. Найдите субдифференциал $\partial \lambda_{\max}(X)$. Здесь $\mathbb{S}^d$ - симметричные матрицы.
    
    Указание: воспользуйтесь вариационным представлением $\lambda_{\max}$ и формулой для субдифференциала максимума.
    
    \item (1.5 балла) Покажите, что функция $\lambda_{\max}(X)$ дифференцируема в точке $X \in \mathbb{S}^d$ тогда и только тогда, когда максимальное собственное значение матрицы $X$ является простым (т. е. имеет кратность 1). Чему равен градиент $\nabla \lambda_{\max}(X)$?
\end{enumerate}


\end{document}